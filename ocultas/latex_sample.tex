\documentclass{article} % define el tipo de documento, y por lo tanto el formato que seguir� el compilador por defecto
\usepackage[utf8]{inputenc} % fuente del texto
\usepackage{subfiles} % para incluir texto de otros archivos .tex
\usepackage[spanish]{babel} % idioma de fechas, secciones, �ndices, t�tulos, etc.
\usepackage{amsmath} % mejora el formato de Math Mode, permite ecuaciones align, etc.
\usepackage{amssymb} % permite utilizar una variedad de s�mbolos adicionales
\usepackage{amsfonts} % mejora el formato de Math Mode
\usepackage{esvect} % vectores m�s largos con \vv
\usepackage{siunitx} % unidades sistema internacional
\usepackage{graphicx} % figuras
\usepackage{subcaption} % subfiguras
\usepackage[colorlinks=true, linkcolor=blue]{hyperref} % referencias con enlaces, se define el color o "false" si se quieren enlaces negros
\usepackage{nomencl} % para lista de abreviaturas
\makenomenclature % para lista de abreviaturas

% Margenes del documento
\oddsidemargin 0cm
\textwidth 16.5cm
\textheight 21cm
\topmargin -1cm

\title{Clase de Introducci�n al \LaTeX}
\author{Asociaci�n Aeroespacial Cosmos}
\date{\today} % proporciona la fecha autom�ticamente. Ojo con las entregas atrasadas :P

\begin{document}

\maketitle
\tableofcontents

\section{Insertar archivos}
Estas dos l�neas est�n comentadas ya que, si no existe el archivo deseado, el documento no compilar�. Ver c�digo fuente. Comenta o descomenta mucho texto a la vez con ctrl + \textbackslash
% \input{archivo.tex} % insertar otros archivos de texto de la carpeta general
% \input{folder/archivo.tex} % insertar archivos de texto de otra carpeta

\section{Ejemplo de Section}
\subsection{Ejemplo de Subsection}
\subsubsection{Ejemplo de Subsubsection}
\paragraph{Ejemplo de Paragraph}
\section*{Ejemplo de Section sin n�mero}
Para obtener secciones, figuras, ecuaciones, etc. sin n�mero, a�ade un asterisco entre el nombre y el primer corchete.

\section{Listas}
Este es un ejemplo de una lista sin enumerar:
\begin{itemize}
    \item item 1
    \item item 2
        \begin{itemize}
            \item esto es una lista
            \item dentro de una lista
        \end{itemize}
    \item item 3
\end{itemize}

\bigskip % esto a�ade espacio vertical, as� como \smallskip
Este es un ejemplo de una lista enumerada:
\begin{enumerate}
    \item item 1
    \item item 2
        \begin{enumerate}
            \item esto es una lista
            \item dentro de una lista
        \end{enumerate}
    \item item 3
        \begin{itemize}
            \item esto es una lista sin enumerar
            \item dentro de una lista enumerada
            \item se pueden mezclar :)
        \end{itemize}
\end{enumerate}

\section{Math Mode}
Algunos ejemplos son letras griegas $\alpha = \omega$, o fracciones: $$ \frac{x}{3r} $$

Integrales, par�ntesis, subscripts y superscripts (en modo equation):
\begin{equation}
    \int_{x_o}^{x(a)} \frac{\partial x}{a} = 3x \left( V_{20}^M \right)
\end{equation}
recuerda que para eliminar el n�mero de la ecuaci�n se puede a�adir un asterisco, tanto en \textit{begin} como en \textit{end}.

Ejemplos en modo align, con \& para centrar:
\begin{align}
    \vv{V}_{21}^C &= \vv{V}_{21}^I + \vv{\omega}_{21} \times \vv{\text{IC}} \\
    \vv{V}_{21}^C &= \enspace0\; + \vv{\omega}_{21} \times \vv{\text{IC}} \nonumber % esto elimina el n�mero de la ecuaci�n
\end{align}

Ejemplo de ecuaciones:
\begin{equation}
    \vv{\alpha}_{21} = \omega \left(\frac{2+\sqrt{3}}{3}\right) \omega \left(\frac{2\sqrt{3}+3}{6}\right)
        \begin{array}{|ccc|}
             \vv{\imath}_1 & \vv{\jmath}_1 & \vv{k}_1\\
             0 & -1 & \sqrt{3} \\
             0 & 0 & 1 \\
        \end{array}
\end{equation}

\begin{equation}
    \begin{array}{ccl}
        \vv{\imath}_0 &=& \enspace\;\cos{\psi}\:\vv{\imath}_1 + \sin{\psi}\:\vv{\jmath}_1\\
        \vv{\jmath}_0 &=& -\sin{\psi}\:\vv{\imath}_1 + \cos{\psi}\:\vv{\jmath}_1\\
        \vv{k}_0 &=& \vv{k}_1
    \end{array}
    \label{eq:versores0from1}
\end{equation}

\clearpage % este es un pagebreak, obviously

\section{Cuadros y figuras}
\begin{table}[h!]
    \centering
    \caption{Longitudes medidas en laboratorio.}
    \label{cuadro:longitudes}
    \begin{tabular}{l|cc} % centrado de cada columna, l�meas verticales
         \textbf{Tuber�a} & \textbf{Longitud0 [m]} & \textbf{Diametro} [m] \\
         \hline % horizontal line
         Inlet 1          & 3                & 2               \\
         Inlet 2          & 1                & 0.2         
    \end{tabular}
\end{table}

Ejemplo de array (Math Mode, tiene que estar dentro de una ecuaci�n).
\begin{equation}
    \label{cuadro:variables}
    \begin{array}{ccc}
         \textbf{Variable} & \textbf{Valor} & \textbf{Coeficiente} \\
        \hline
        a         & \zeta            & c         \\
        x         & \nu              & 3K
    \end{array}
\end{equation}

\begin{figure}[h!]
    \centering
    \includegraphics[width=0.5\textwidth]{figura_ejemplo.png}
    \caption{Ejemplo de una figura}
    \label{fig:figura_de_algo}
\end{figure}

\begin{figure}[h!]
    \centering
    \begin{subfigure}[b]{0.4\textwidth}
        \centering
        \includegraphics[width=\textwidth]{figura_ejemplo.png}
        \caption{Subcaption 1}
        \label{fig:subfigure_example_a}
    \end{subfigure}
    \quad % esto a�ade un poco de espacio entre figuras
    \begin{subfigure}[b]{0.4\textwidth}
        \centering
        \includegraphics[width=\textwidth]{figura_ejemplo.png}
        \caption{Subcaption 2}
        \label{fig:subfigure_example_b}
    \end{subfigure}
    \caption{Ejemplo de una figura con subfiguras.}
    \label{fig:figure_with_subcaptions}
\end{figure}

N�tese como queda el �ndice al referenciar una subfigura (como, por ejemplo, \autoref{fig:subfigure_example_a}).

\section{Bibliograf�a y citaciones}
Esto es un ejemplo de una citaci�n \cite{rocketdyne}. Esto es otro ejemplo (\cite{thermal_engs}) de una citaci�n. Para compilar la bibliograf�a, insertar el comando al final del documento.

\section{Nomenclatura}
Se definen las variables del siguiente modo, y la lista aparece donde de llame (makenomenclature), en este caso al final del documento.
\nomenclature{$\rho$}{Densidad}
\nomenclature{$S_m$}{Masa a�adida al sistema por la fuente}
\nomenclature{$\Vec{v}$}{Velocidad}
\nomenclature{$\overline{\overline{\tau}}$}{Tensor de tensiones}


%%%%%%%%%%%%%%%%%%%%%%%%%%%%%%%%%%%%%%%%%%%%%%%%%
% al cambiar el idioma del documento, los t�tulos de estas secciones tambi�n cambian:

% Bibliograf�a
\bibliographystyle{unsrt}
\bibliography{refs.bib}

% Nomenclatura o lista de abreviaturas
\printnomenclature[0.5in]

% �ndices de figuras y cuadros
\listoftables
\listoffigures
%%%%%%%%%%%%%%%%%%%%%%%%%%%%%%%%%%%%%%%%%%%%%%%%%%


\end{document}

@inbook{rocketdyne,
author = {Vince Wheelock  and  Robert Kraemer },
year = {2006},
month = {},
pages = {},
title = {Rocketdyne: Powering Humans into Space},
publisher = {American Institute of Aeronautics and Astronautics, Inc.},
isbn = {978-1-56347-754-6},
doi = {10.2514/4.477546}
}

@inbook{thermal_engs,
author = {Oleg N. Favorsky},
year = {2009},
month = {12},
pages = {},
title = {Thermal to Mechanical Energy Conversion :Engines and Requirements - Volume II},
publisher = {Eolss Publishers Co. Ltd.},
isbn = {978-1-84826-022-1},
}